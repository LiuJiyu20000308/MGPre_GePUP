
\section{Time Integration}
\label{sec:TimeIntegration}

We apply ERK-ESDIRK to semi-discrete formulae \eqref{eq:SemiGePuP}, in
which the RHS of \eqref{eq:SemiGePuPa} is divided into an explicit
part and an implicit part:
\begin{subequations}
  \begin{eqnarray}
    &\mathbf{X}^{[E]} = \left<\mathbf{g}\right>-\mathbf{D}\left<\mathbf{uu}\right>
      - \mathbf{G}\left<q\right>, \\
    & \mathbf{X}^{[I]} = \nu\mathbf{L}\left<\mw\right>.
  \end{eqnarray}
\end{subequations}

Now we obtain GePuP-IMEX algorithm:
\begin{subequations}
  \begin{eqnarray}
    &&\left<\mathbf{w}\right>^{(1)}=\left<\mathbf{w}\right>^{n}, \\
    &&\left\{
    \begin{array}{l}
      \mbox{for } s=2,3,\cdots,n_s, \\
      \left(\mathbf{I}-\Delta
      t\gamma\nu\mathbf{L}\right)\left<\mathbf{w}\right>^{(s)} =
      \left<\mathbf{w}\right>^{n} + \Delta
      t\sum\limits_{j=1}^{s-1}a_{s,j}^{[E]}\mX
      ^{[E]}\left(\left<\mathbf{u}\right>^{(j)},t^{(j)}\right) +\Delta
      t\nu\sum\limits_{j=1}^{s-1}a_{s,j}^{[I]}\mL\left<\mathbf{w}\right>^{(j)},\\
      \left<\mathbf{u}\right>^{(s)}=\mathbb{P}\left<\mathbf{w}\right>^{(s)},
    \end{array} 
    \right.\\
    &&\left\{
       \begin{array}{l}
         \left<\mathbf{w}\right>^*=\left<\mathbf{w}\right>^{(n_s)}+\Delta
         t\sum\limits_{j=1}^{n_s}\left(b_j-a_{n_s,j}^{[E]}\right)\mX^{[E]}\left(\left<\mathbf{u}
         \right>^{(j)},t^{(j)}\right),\\
         \left<\mathbf{u}\right>^{n+1}=\mathbb{P}\left<\mathbf{w}\right>^*,\\
         \left<\mathbf{w}\right>^{n+1}=\left<\mathbf{u}\right>^{n+1}.
       \end{array}
       \right.
  \end{eqnarray}
\end{subequations}

There are three linear systems of equations need to be solved at each
intermediate stage:
\begin{enumerate}
\item A Poisson equation with Neumann boundary condition for
  extracting $\left<q\right>$ from $\left<\mathbf{u}\right>$.
\item A Helmholtz equation with no-slip boundary condition for
  evaluating $\left<\mathbf{u}\right>$.
\item A Poisson equation with Neumann boundary condition for
  projecting $\left<\mw\right>^*$.
\end{enumerate}

Above all, the essence of our work is solving a Poisson equation
$L\phi=b$ on
an irregular domain, the spatial discretization is  the same as that
in section \ref{sec:SpatialDiscretiazation}.

%%% Local Variables:
%%% mode: latex
%%% TeX-master: "MGPreconditioner_MathDoc"
%%% End:
